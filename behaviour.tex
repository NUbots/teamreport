\section{Behaviour}
\label{Behaviour}
\subsection{Strategy}
Moving to the Nao League, we aimed to carry over the overall strategies that had proved successful for us in previous years of the Aibo League. Our strategy involved Trying to find the ball as quickly as possible, moving as fast as possible towards the ball when it is seen, and reducing the amount of wasted time when at the ball by only reducing speed when absolutely required. This strategy works well when you can move at an equal or greater speed to your opponents. The basis behind this is that if you can get to the ball faster than your opponent you get much more possession. You are therefore able to keep control of the ball and give your opponents less chances to score against you.\\

Upon development of our walk we found that the ball would be effectively `kicked' when the robot would walk into it. This played well into our overall strategy as it meant that we would not need to stop and kick the ball every time we wanted to move it, greatly reducing our time spent manipulating the ball. There are a few advantages to using this method. It allows the robot to continue moving at full speed. Also once the ball has been `kicked' the robot is already moving towards its final destination, effectively giving you a head start on your opponents. In this way our motion was more like a form of dribbling in regular soccer than a shot. The major downside was that it was difficult to control the direction of the ball, with the only controlling factor being the direction of approach to the ball which still would not give an accurate angle depending on the manner in which the foot contacted the ball. This randomness both helped and hurt us during the competition, leading to goals both for and against.\\

\subsection{Searching}
To find the ball we used a simple searching method. The robot would pan its head twice from side to side, one pan with the robots head relatively level, the second with its head looking down. Following this the robot would then turn approximately 90 degrees. This was repeated 4 times to complete a  full circle the robot then taking a short walk to the next search location. This movement of location was to account for balls that were close to the robot, but could not be seen without bending over excessively.\\

\subsection{Chasing}
Chasing was used as the core component to our behaviour. Once a ball is seen the relative distance and bearing of the ball from the robot was used to calculate an arced walk of four steps. These four step arcs are generated as the previous one is completed. As the robot approached the ball it enters a region in which it cannot track the ball using only its head. Upon entering this region the robot would move over the position of the last seen ball. This results in a `kick' and the robot would hopefully then be able to see the ball again and continue with its next chase.\\ 

To add some intelligence to this behaviour, the walk arc was calculated not to move to the ball, but to a position generated using the positions of the robot, the ball and the goals. The aim was to approach the ball while moving towards the opponents goals, thereby moving the ball towards the opponents' goals and away from our own.\\