\section{Vision}
The Nao vision system followed the process: \emph{Input image, classify image, form blobs, sort colour blobs, combine soft blobs, find ball, find goals, line detection}. Even though the vision process was based on NUbot code the only section that could be reused was blob formation. 

The 2008 vision module was very simple. This section details the main functions of the vision system and the reasons behind them. 
\subsection{Overview}

Webots was used to delevop the base vision system. Inputting the rgb image and classifying it using a perfect look-up table (LUT) allowed blob formation and basic object recognition to be tested which then allowed for localisation development to begin.

We then worked on developing the vision debug application \emph{Eye of the NUManoid} (EOTN). What made this application different is the HSI classifcation method. Transforming the YUV image to HSI values and selecting the classification region in HSI allowed for an intuitive classification method and better separation of hue ranges. 

Once we had a robot we were able to select camera settings using \emph{Telepathe} and save image streams. We could then work on object recognition. It was very basic, however localisation was quite basic aswell. 

%\emph{Input image, classify image, form blobs, sort colour blobs, combine soft blobs, find ball, find goals, line detection}


\subsection{Image Quality and Camera Settings}
%One good thing about the transition from the Aibo to the Nao was the improvement in camera quality. While the camera still had its share of problems, such as random swaps of U and V channels, random blackouts and random ghost camera settings, at least it made the separation of colours easier.
%The Nao has three different resolutions for which we could obtain images, 160x120, 320x240 and its native resolution 640x480. 
This year was the first year where we successfully started processing larger image resolutions. We increased the image resolution from 160x120 pixels to double the resolution, 320x240 pixels. This enabled clearer images and more accurate distances when we found recognisable objects. For example a ball could be detected by the goal keeper standing at its goal from the other end of the field. However, with higher resolution images the calculations to process an image became computationally heavy. This forced us to reduce our frame rate to 15 frames per second.

There was an additional camera installed around the nose of the Nao. This camera was particularly useful for lining up kicks. However, switching over from one to another would take time, which equates to the loose of a few frames. Logic was inserted in the search behaviour to reduce the number of camera transitions. 

\subsubsection{Camera settings}

The Nao camera has 6 main camera settings with a large range of values for each setting. This means that the camera settings are very sensitive to lighting changes and not only do colour tables need to be recalibrated for any change, but also the camera settings. At the RoboCup venue the camera settings and colour tables had to be recalibrated for both the main field and test fields. 

The robot is mobile with a movable head. This means that images are subject to blurring and lighting flare. Camera settings are chosen to minimise blurring and allow for best colour separation.

\subsection{Colour Classification}

Colour classification is the top tier in the software system. Any error in classification has a flow on effect to the entire system. Image pixels are mapped to a colour class by referencing a LUT. The environment of the Standard Platform League is colour coded so that objects are recognisable by colour. Rather than classify pixels based on object colour classes alone, additional `soft' colour classes are used to allow additional colour information to be classified and to reduce the risk of false positive classification.

\subsubsection{Soft Colour Classification}

Soft colour classification is used to classify regions of colour that belong to an object but is less saturated due to lighting conditions or overlaps with another colour class. This allows the decision about what object the soft colour belongs to, to be delayed until the entire image is processed.

Using soft colour allows for the classification of `noisy' shades of colour and allows valuable object size information to be kept. The risk of false positives is reduced by allowing true colour to be classed as the shades of colour that are highly saturated. Object decisions are based on the presence of true colour and then soft colour can be used for additional size information.

The shiny bright blue uniforms of the Nao have created a significant overlap of colour values with the blue goal. Similarly, the shiny red uniforms become less saturated with the reflection of light and the hue values shift closer to orange.

Soft colour is also applied to less saturated shades of colour. These shades are more likely to occur in background noise. Soft colour is used to classify additional bright and dark shades of red, orange, yellow and blue. 
 
Blobs are formed for each colour class, including soft colours. Once the image is processed soft colour decisions are made based on blob arrangement in Soft Colour Filtering. 

\subsection{Blob Formation}

Grouping of classified colours is used for the transfer of information from an image to object recognition. When forming blobs the colour, size and area information is required. Blobs are formed based on classified colour of the classified image. Undefined, white and field green and the soft colour shadow-black are not included in blob formation. This is due to the fact that size and shape information is not required for objects of these colours. It is also due to the nature of distribution of these colours. They are all abundantly spread throughout images and scattered; forming blobs on images would involve unnecessary processing. 

Blob formation checks every pixel of the image when forming blobs, this means we can form blobs as small as 1x1, however, we only use blobs greater than or equal to 3x3 pixels. 

\subsection{Sort Colour Blobs}

Soft colour blobs are filtered and kept if of sufficient size or overlapping with the corresponding true colour. During this stage, a simple check is also performed on the ball object. The ball must sit below the highest green transition (a simple green horizon) to be a valid ball.

\subsection{Combine Soft Blobs}

Overlapping soft colour blobs are merged. Colour blob clusters are then made to prepare information for object recognition. Cluster variables include: exists, correct pixels, colour 2 correct pixels, min x, max x, min y, max y, height, width, area. 
Yellow blobs, shadow yellow blobs and all yellow blobs are clustered for yellow goal recognition. Blue blobs, shadow blue blobs and all blue blobs are clustered for blue goal recognition and blue robot separation. Red blobs and shadow red blobs are clustered for red robot detection.

Clustering allows object recognition to have a region of interest. With the use of soft colour most goal images are made of multiple blobs.
\subsection{Ball}

Ball recognition was based on previous code. The main difference is that we don't have to deal with the upclose ball situation. Instead we have to be able to see the ball at an elevated position and with additional glare shining off the ball. A confidence system is used to decide the ball. Confidence is weighted based on correct pixel to size ratio, size and circle fit.

\subsubsection{Ball Distance, Bearing and Elevation Calculations}
The raw values of bearing and elevation are calculated using the centre x and centre y of the blob in the image, while the raw
distance is calculated using the width of the blob in pixels. These resulting values are relative to the camera. They are then transformed using the forward kinematics of the robot to give a relative location in terms of the robots reference frame.

\subsubsection{Circle Fitting}

Least squares circle fitting has been implemented to improve the distances, bearings and elevations of balls that are not fully
visible in the image. There are two main parts to the circle fitting procedure. Firstly the selection of the points to be fit, followed by the fitting of a circle to these points. These points are gathered during one of various scan types depending on the positioning of the ball in the image. The points are found by searching from the outside of the blob inwards until it finds pixels of the same colour of the blob, or in the case of the orange blobs also one of the soft colours close to orange. The directions for which scans have been created are: \emph{Left to Right, Right to Left, Top to Bottom, Bottom to Top,Simultaneous Left to centre \& Right to centre and Simultaneous Top to Centre \& Bottom to Centre}. These can be seen in Figure~{\ref{fig:objectBall3}. The reason the different scanning directions are needed is so that in images in which the view of the ball is cut off by the edge of the image, points are not selected along the edge of the image in turn biasing the circle fit. For more information on the least squares fitting function used see \cite{Seysener2003},\cite{SeysenerMurchMiddleton2004}.

\begin{figure}[!ht]
\begin{center}
    %\leavevmode
    \scalebox{0.3} {\includegraphics{stevenfigs/objectBall3.png} }
    \caption{Ball scanning directions,  (top left) left to right, (middle left) right to left, (bottom left) left and right to centre, (top right)  top to bottom, (middle right) bottom to top, (bottom right) top and bottom to centre.}
    \label{fig:objectBall3}
\end{center}
\end{figure}

Once a circle has been fit to these points, the validity of the fit circle is determined. If the diameter of the circle is significantly lower that the width of the blob then it is assumed that the circle fit is not correct.


\subsection{Goals}

With the use of soft colour classification and due to obstructions on the field, the goal is often not seen as a single blob, but rather a cluster of blobs. A scan method was used on clusters to detect the gap type and calculate the width of posts.

Gap types are: no gap (unknown post), left (right post), right (left post), middle (both posts, no crossbar), bottom (both posts and crossbar) or undefined. 

Three main variables were used to calculate goal confidence. Correct pixel to size ratio, cluster height to width ratio and goal gap detected.

 
\subsubsection{Goal Distance, Bearing and Elevation Calculations}

The raw distance of the goal is found by using the width of a post. With the new tall goals and the robot camera position looking down for a ball, the goal height is unreliable. If both goal posts are seen both widths are used to calculate the distance to each post. 

The raw bearing and elevation are found using the centre of the post. The method to find the center of the post varies depending on the gap type. As with the ball these raw distances are translated back so they are relative to the hips of the robot. The elevation is not very reliable as the top of the post is often not seen.

\subsubsection{Robots}

Red robots are detected by clusters of red blobs.





\subsection{Line Detection}

The detection of field lines is approached in a completely different manner to that of other Objects. When searching for objects, sections of a specific colour are joined together to form a blob. While this works well for most objects, white causes problems due to its abundance in the image and hence is normally ignored. Also the thin nature of lines, and the robot's low Point Of View, makes missing pixels very common within the classified image. Finally the storing of blobs, as a square which bounds the entire object, does not store enough important information about the line.

To over come these issues the detection of lines is based around the two unique features that field lines have; their long thin length and their green-white-green transitions. Using these two details the image can be sparsely searched, since the search line does not need to find every point of the line, just enough to re-create the line. Also the transition information allows many white pixels to be thrown out early in the detection process, thereby reducing the over all load on the processor. This reliance on points and not the entire line also allows the lines to be partially hidden behind another object without effecting their detection.

With these factors in mind, the image can now be efficiently searched in the following steps;
\begin{itemize}

\item The image is searched using a horizontal and vertical search grid. The search is restricted to the area in the image below the horizon line and picks out points of sharp contrast which have green-white then white-green transitions. These points are recorded in an array for later use.

\item The found points are then checked against each other to form possible lines. Once candidates are found, more points are checked and added until a line is formed. All checks on the lines at this stage is done based on gradient to allow the fastest line formation.

\item The lines are cleaned up to make sure all points actually fit on the line. Lines which have too large a number of points not contained within the final line are removed.

\item Lines are checked against each other to confirm that they are not just segments of larger lines. The allows lines to be formed that have a break in the middle, such as when a robot is on top of the line.

\item Corner points are found by extending lines and finding their intersection locations. The intersections are checked to confirm virtual points are not found. 

\item An attempt is made to uniquely identify corner points from other objects within the image. While this is not always needed, if a unique identification is made the use of the corner point within localisation becomes much more efficient.

\end{itemize}

\begin{figure}[htpb]
\begin{center}
    %\leavevmode
    \scalebox{0.8} {\includegraphics{RobinFig/linesfig.png} }
    \caption{Line detection steps. (top left) original image. (top right)Edge Classified image with vertical scan lines. (bottom left) the found line points. (bottom right) Final lines with the corner uniquely identified.}
    \label{fig:lines1}
    %\leavevmode
\end{center}
\end{figure}

For more details on the exact methods used please refer to the 2005 team report.